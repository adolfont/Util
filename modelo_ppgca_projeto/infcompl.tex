\chapter{Informa\c{c}\~oes Complementares}
\label{chap:infcompl}

% Trecho extraído do modelo http://www2.dainf.ct.utfpr.edu.br/ppgca/estrutura-academica/regulamentacao/avaliacao/copy_of_ModeloProjetoeSeminarios.doc/view
Neste capítulo constará uma série de informações que se destinam a subsidiar os membros das bancas do Projeto de Dissertação de Mestrado e dos Seminários de Acompanhamento I e II. 
Na versão final da dissertação este capítulo deverá desaparecer e, apresentando alguma informação pertinente e necessária ao conteúdo da dissertação, esta deverá ser transferida para o local apropriado na dissertação.

\section{Or\c{c}amento}
Aqui deve ser apresentada toda a relação de material permanente e de consumo que será utilizado no projeto, a sua quantidade, o seu custo (caso seja gratuito, indicar a gratuidade) e o financiador. Caso estes materiais, permanentes ou de consumo, e equipamentos não estejam disponíveis na Universidade Tecnológica Federal do Paraná (UTFPR), o pesquisador deve informar como irá obtê-los ou a entidade que os possui e que possa disponibilizá-los para a sua pesquisa. A aprovação do Projeto de Dissertação de Mestrado, ou dos Seminários de Acompanhamento I ou II, não garante nenhuma forma de financiamento adicional.
Materiais de consumo que fazem parte do arsenal trivial de um laboratório ou serviço devem ser também listados e orçados, independentemente da quantidade a ser utilizada.

\section{Dificuldades encontradas}
Aqui devem ser descritas as dificuldades encontradas na realização no trabalho, bem como os caminhos adotados para superá-las.

\section{Etapas e Cronograma}
Apresentação através de texto, tabela, planilha ou esquema, da distribuição das várias etapas do projeto ao longo do período previsto para sua execução. O cronograma deverá permitir uma visão ampla do projeto, de seus objetivos, e suas etapas, facilitando a identificação das atribuições de todos os participantes do projeto.

\subsection{Etapas}
Sugere-se que o cronograma seja organizado em etapas conforme o seguinte modelo:

\begin{table}[htbp]
	\centering
	\begin{tabular}{|c|p{8cm}|}
		\hline
		NOME DA ETAPA & DESCRIÇÃO DA ETAPA \\ \hline
		Etapa 1 & Descrição da etapa  \\ \hline
		Etapa 2 & Descrição da etapa  \\ \hline
		... & ... \\ \hline
		Etapa n & Descrição da etapa n \\ \hline
	\end{tabular}
	\caption{Etapas do Projeto}
	\label{quadro:etapas}
\end{table}

O número de etapas varia conforme o projeto. Caso haja variação entre as etapas apresentadas em documentos anteriores, as diferenças devem ser apresentas também.

\subsection{Cronograma mensal das etapas de desenvolvimento do trabalho}

Deverão ser apresentadas as tarefas programadas e as realizadas no momento da entrega do Pré-Projeto de Dissertação de Mestrado ou do Projeto de Dissertação de Mestrado. A seguir é apresentada uma sugestão de representação de cronograma. Caso seja necessário, o quadro deve ser dividido.


\begin{table}[htbp]
	\begin{tabular}{|p{3cm}|p{0.07cm}|p{0.07cm}|p{0.07cm}|p{0.07cm}|p{0.07cm}|p{0.07cm}|p{0.07cm}|p{0.07cm}|p{0.07cm}|p{0.07cm}|p{0.07cm}|p{0.07cm}|p{0.07cm}|p{0.07cm}|p{0.07cm}|p{0.07cm}|p{0.07cm}|p{0.07cm}|p{0.07cm}|p{0.07cm}|p{0.07cm}|p{0.07cm}|p{0.07cm}|p{0.07cm}|p{0.07cm}|}
		\hline
		\multicolumn{ 2}{|l|}{\textbf{Ano}} & \multicolumn{ 10}{c|}{\textbf{200...}} & \multicolumn{ 12}{c|}{\textbf{200...}} & \multicolumn{ 2}{c|}{\textbf{200...}} \\ \hline
		\multicolumn{ 2}{|l|}{\textbf{Etapas / Mês}} & \textbf{M} & \textbf{A} & \textbf{M} & \textbf{J} & \textbf{J} & \textbf{A} & \textbf{S} & \textbf{O} & \textbf{N} & \textbf{D} & \textbf{J} & \textbf{F} & \textbf{M} & \textbf{A} & \textbf{M} & \textbf{J} & \textbf{J} & \textbf{A} & \textbf{S} & \textbf{O} & \textbf{N} & \textbf{D} & \textbf{J} & \textbf{F} \\ \hline
		Nome da etapa 1 & P & \textbf{X} & \textbf{X} & \textbf{X} & \textbf{} & \textbf{} & \textbf{} & \textbf{} & \textbf{} & \textbf{} & \textbf{} & \textbf{} & \textbf{} & \textbf{} & \textbf{} & \textbf{} & \textbf{} & \textbf{} & \textbf{} & \textbf{} & \textbf{} & \textbf{} & \textbf{} & \textbf{} & \textbf{} \\ \hline
		\textbf{} & \textbf{R} & \textbf{X} & \textbf{X} & \textbf{} & \textbf{} & \textbf{} & \textbf{} & \textbf{} & \textbf{} & \textbf{} & \textbf{} & \textbf{} & \textbf{} & \textbf{} & \textbf{} & \textbf{} & \textbf{} & \textbf{} & \textbf{} & \textbf{} & \textbf{} & \textbf{} & \textbf{} & \textbf{} & \textbf{} \\ \hline
		Nome da etapa 2 & P & \textbf{} & \textbf{X} & \textbf{X} & \textbf{X} & \textbf{} & \textbf{} & \textbf{} & \textbf{} & \textbf{} & \textbf{} & \textbf{} & \textbf{} & \textbf{} & \textbf{} & \textbf{} & \textbf{} & \textbf{} & \textbf{} & \textbf{} & \textbf{} & \textbf{} & \textbf{} & \textbf{} & \textbf{} \\ \hline
		\textbf{} & \textbf{R} & \textbf{} & \textbf{X} & \textbf{X} & \textbf{X} & \textbf{} & \textbf{} & \textbf{} & \textbf{} & \textbf{} & \textbf{} & \textbf{} & \textbf{} & \textbf{} & \textbf{} & \textbf{} & \textbf{} & \textbf{} & \textbf{} & \textbf{} & \textbf{} & \textbf{} & \textbf{} & \textbf{} & \textbf{} \\ \hline
		... & P & \textbf{} & \textbf{} & \textbf{} & \textbf{} & \textbf{} & \textbf{} & \textbf{} & \textbf{X} & \textbf{X} & \textbf{X} & \textbf{} & \textbf{} & \textbf{} & \textbf{} & \textbf{} & \textbf{} & \textbf{} & \textbf{} & \textbf{} & \textbf{} & \textbf{} & \textbf{} & \textbf{} & \textbf{} \\ \hline
		\textbf{} & \textbf{R} & \textbf{} & \textbf{} & \textbf{} & \textbf{} & \textbf{} & \textbf{} & \textbf{} & \textbf{} & \textbf{X} & \textbf{X} & \textbf{} & \textbf{} & \textbf{} & \textbf{} & \textbf{} & \textbf{} & \textbf{} & \textbf{} & \textbf{} & \textbf{} & \textbf{} & \textbf{} & \textbf{} & \textbf{} \\ \hline
		Nome da etapa n & P & \textbf{} & \textbf{} & \textbf{} & \textbf{} & \textbf{} & \textbf{} & \textbf{} & \textbf{} & \textbf{} & \textbf{} & \textbf{} & \textbf{} & \textbf{} & \textbf{} & \textbf{} & \textbf{} & \textbf{} & \textbf{} & \textbf{} & \textbf{} & \textbf{} & \textbf{} & \textbf{X} & \textbf{X} \\ \hline
		\textbf{} & \textbf{R} & \textbf{} & \textbf{} & \textbf{} & \textbf{} & \textbf{} & \textbf{} & \textbf{} & \textbf{} & \textbf{} & \textbf{} & \textbf{} & \textbf{} & \textbf{} & \textbf{} & \textbf{} & \textbf{} & \textbf{} & \textbf{} & \textbf{} & \textbf{} & \textbf{} & \textbf{X} & \textbf{X} & \textbf{} \\ \hline
	\end{tabular}
	\caption{Cronograma. P: Programado; R: Realizado}
	\label{tab:cronograma}
\end{table}

%-------------------------------------------------------------------------------