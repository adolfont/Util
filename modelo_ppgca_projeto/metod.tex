\chapter{Metodologia}
\label{chap:metod}

% Trecho extraído do modelo http://www2.dainf.ct.utfpr.edu.br/ppgca/estrutura-academica/regulamentacao/avaliacao/copy_of_ModeloProjetoeSeminarios.doc/view
Outros títulos muitos comuns para esse capítulo são MÉTODOS ou MATERIAIS E MÉTODOS.
É uma descrição técnica de como será desenvolvido ou foi desenvolvido o trabalho. Devem estar detalhadas, de forma lógica, linear e cronológica, todas as etapas do projeto.
Uma metodologia bem estruturada reflete um bom planejamento do processo de investigação, diminuindo a possibilidade de surgirem falhas que impeçam a conclusão do projeto.
A metodologia contempla, entre outros: como será feito o levantamento bibliográfico, indicando as áreas a serem estudadas e critérios de inclusão e exclusão da literatura; o tipo do estudo; o local onde será desenvolvido; a população, a amostra selecionada e os critérios adotados; a coleta de dados (instrumentos e procedimentos de coleta); o desenvolvimento do aparato tecnológico em questão, como, por exemplo, software ou hardware; análise dos dados; aspectos éticos envolvidos na pesquisa. Os modelos de questionários, entrevistas e materiais complementares utilizados podem ser apresentados nos resultados ou em apêndices, quando de autoria do aluno, ou em anexos, quando de autoria de terceiros.
Eventualmente, durante a descrição, serão necessárias justificativas para a escolha de um ou outro método, e, mesmo que o projeto proponha uma metodologia inédita, as referências bibliográficas devem ser apresentadas.
A abordagem que será utilizada para a análise dos resultados também deve ser explicitada, indicando o teste estatístico ou processo analítico que permitirá a extração de conclusões.
É importante deixar bem claro o processo de avaliação e validação dos resultados a serem obtidos. Não basta apenas dizer que o será avaliado, sendo necessário descrever detalhadamente todo o processo de avaliação, bem como descrever o processo de validação.
A metodologia é que efetivamente demonstra o caminho selecionado e trilhado pelo pesquisador para materializar o trabalho e atingir os objetivos propostos, devendo, portanto, ser clara e detalhadamente descrita.
Devem ser descritas as alterações entre a metodologia apresentada no Projeto de Dissertação de Mestrado e nos Seminários de Acompanhamento I e II (caso hajam).
%-------------------------------------------------------------------------------