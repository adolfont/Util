\chapter{Lista de siglas}
\label{chap:siglas}

% Trecho extraído do modelo http://www2.dainf.ct.utfpr.edu.br/ppgca/estrutura-academica/regulamentacao/avaliacao/copy_of_ModeloProjetoeSeminarios.doc/view
Elemento opcional, constituída de uma relação alfabética das abreviaturas e siglas utilizadas no texto, seguido das palavras ou expressões correspondentes grafadas por extenso. Quando necessário, recomenda-se a elaboração de lista própria para cada tipo. Caso não haja abreviaturas, não é necessário incluir a lista de abreviaturas. 

(Adaptado para usar o comando \\sigla)

\begin{itemize}
	\item \sigla{Bps}{bits por segundo}: bits por segundo
	\item \sigla{CGI}{Common Gateway Interface - Interface de Porta Comum}: Common Gateway Interface - Interface de Porta Comum
	\item \sigla{CNS}{Cartão Nacional de Saúde}: Cartão Nacional de Saúde
\end{itemize}

%-------------------------------------------------------------------------------

\textbf{* Observa\c{c}\~oes:} a lista de siglas nao realiza a ordenacao das siglas em ordem alfabetica
 Em breve isso sera implementado, enquanto isso: \\
\textbf{Sugest\~ao:} crie outro arquivo .tex para siglas e utilize o comando \textbackslash sigla\{sigla\}\{descri\c{c}\~ao\}.
Para incluir este arquivo no final do arquivo, utilize o comando \textbackslash input\{arquivo.tex\}.
Assim, Todas as siglas serao geradas na ultima pagina. Entao, devera excluir a ultima pagina da versao final do arquivo
 PDF do seu documento.
