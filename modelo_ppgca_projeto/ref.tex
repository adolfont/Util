\chapter{Refer\^encias}
\label{chap:ref}

% Trecho extraído do modelo http://www2.dainf.ct.utfpr.edu.br/ppgca/estrutura-academica/regulamentacao/avaliacao/copy_of_ModeloProjetoeSeminarios.doc/view

Todas as referências citadas no texto devem estar relacionadas em ordem alfabética, conforme os exemplos descritos a seguir, que seguem as normas da ABNT. Não devem ser apresentadas referências que não foram citadas no texto.
A referência é um conjunto padronizado de elementos descritivos, retirados de um documento, que permitem sua identificação individual (ASSOCIAÇÃO BRASILEIRA DE NORMAS TÉCNICAS, 2000). As referências devem estar alinhadas à margem esquerda do texto, utilizando espaço simples e separadas umas das outras por dois espaços simples (ASSOCIAÇÃO BRASILEIRA DE NORMAS TÉCNICAS, 2000).
A seguir estão representados alguns exemplos para referências bibliográficas. As normas da ABNT podem ser consultadas em site específico da Biblioteca da UTFPR: \\ \url{http://www.utfpr.edu.br/curitiba/biblioteca-e-producao-academica/ normas-para-elaboracao-de-trabalhos-academicos} . 

LIVROS

SOBRENOME(S) DO(S) AUTORES(ES), Prenome (S) (iniciais ou por extenso). Título da obra: subtítulo. Edição. Local de publicação (Cidade): Editora, data de publicação. Paginação.

Exemplos:

SILVEIRA, I.C. da. O pulmão na prática médica. Rio de Janeiro: Vozes, 1993. 159 p.


LANGE, Danny B.; OSHIMA, Mitsuru. Programming and Deploying Java Mobile Agents with Aglets. Estados Unidos da América: Addison-Wesley, 1998. 227 p.


CAPÍTULO DE LIVRO

SOBRENOME(S) DO(S) AUTOR(ES) da parte referenciada, Prenome (S) (iniciais ou por extenso). Título da parte referenciada. In: SOBRENOMES (S) DO(S) AUTOR(ES) (ou editor, etc), Prenome(s) (iniciais ou por extenso) da publicação. Título da publicação: subtítulo. Edição. Local de publicação (Cidade): Editora, data de publicação. Capítulo, páginas (inicial e final).


Exemplo:

ESPOSITO, G. Os segredos do abismo. In: GOMES, V. A vida abissal. Curitiba: Champanhat, 1932, p. 151-178.


ARTIGOS DE PERIÓDICOS

SOBRENOME(S) DO(S) AUTOR(ES), Prenome (S) (iniciais ou por extenso). Título do artigo: subtítulo. Título da publicação, Local de publicação (Cidade), volume, fascículo, página inicial e final do artigo, periódico e data de publicação.


Exemplos:

MOURA, A.S. de. Direito de habitação às classes de baixa renda. Ciência \& Trópico, Recife, v. 11, n. 1, p. 71-78, 1983.


FERREIRA, Christina Ramires; LOPES, Maria Denise. Complexo hiperplasia cística endometrial/piometra em cadelas – revisão. Clínica veterinária, São Paulo, v. 5, n. 27, p. 36-44, jul. 2000.


MONOGRAFIAS

SOBRENOME(S) DO(S) AUTOR(ES), Prenome (S) (iniciais ou por extenso). Título da obra: subtítulo. Edição. Local de publicação (Cidade): Editora, data de publicação. Paginação.


Exemplos:
GORDON, Richard. A assustadora história da medicina. 5. ed. Rio de Janeiro: Ediouro, 1996. 223 p.


MEGGINSON, Leon C.; MOSLEY, Donald C.; PIETR JR, Paul H. Administração: conceitos e aplicações. 4. ed. São Paulo: Harbra,1998. 614 p.


DISSERTAÇÕES E TESES

SOBRENOME(S) DO(S) AUTOR(ES), Prenome (S) (iniciais ou por extenso). Título da dissertação ou tese: subtítulo. Data (ano de depósito). Folhas. Grau de dissertação ou tese – Unidade onde foi defendida, Local, data (ano da defesa).


Exemplos:

FREITAS, S.R.C. de. Marés gravimétricas: implicações para a placa sul-americana. 1993. 264p. Dissertação (Mestrado em Geofísica) – Instituto Astronômico e Geofísico, Universidade de São Paulo, São Paulo, 1993.


PEIXOTO, Luciano Almeida. Sistema de Apoio à Decisão em Exames Ortopédicos da Coluna Vertebral para Auxílio nos Diagnósticos Fisioterapêuticos das Regiões Cervical e Lombar. 2005. 102p. Dissertação (Mestrado em Tecnologia em Saúde) - Programa de Pós-Graduação em Tecnologia em Saúde, Pontifícia Universidade Católica do Paraná, Curitiba, 2005.


FANTUCCI, I. Contribuição do alerta, da atenção, da intenção e da expectativa temporal para o desempenho de humanos em tarefas de tempo de reação. 2001. 130 p. Tese (Doutorado em Psicologia) – Instituto de Psicologia, Universidade de São Paulo, São Paulo, 2001.


PUBLICAÇÃO DE AUTORIA DESCONHECIDA

pRIMEIRA palavra do título em maiúscula: subtítulo. Edição. Local de publicação (Cidade): Editora, data de publicação. Paginação.


Exemplos:

DESARROLLO energético em América Latina y la economia mundial. Santiago: Ed. Universitária, 1980. 245 p.


Sistema de Informação em Saúde. <http:////www.saude.sc.gov.br// sala\_de\_leitura//artigos//Sistemas\_de\_Informacao//SistemasInformaçãoSaúde.doc>. Acesso em de março de 2001.


ARTIGOS DE JORNAIS

SOBRENOME(S) DO(S) AUTOR(ES), Prenome (S) (iniciais ou por extenso). Título da matéria: subtítulo. Título do Jornal, Local de publicação (Cidade), data de publicação. Seção, caderno ou parte do jornal, página inicial e final do artigo ou matéria.


Exemplos:

SUZUKI JR., M. A melhor de todas as copas. Folha de S. Paulo, 02 jul. 1998. Caderno 4, Copa 98, p. 1.


NAVES, P. Lagos andinos dão banho de beleza. Folha de São Paulo, São Paulo, 28 jun. 1999. Folha Turismo, Caderno 8, p. 13.


NORMAS TÉCNICAS

ORGÃO NORMALIZADOR. Título: subtítulo, número da Norma. Local de publicação (Cidade), data de publicação. Paginação.


Exemplo:

ASSOCIAÇÃO BRASILEIRA DE NORMAS TÉCNICAS. Informação e documentação – referências – elaboração: NBR 6023. Rio de Janeiro, 2000. 356 p.


ARTIGOS DE PERIÓDICOS DISPONÍVEIS EM MEIO ELETRÔNICO

SOBRENOME(S) DO(S) AUTOR(ES), Prenome (S) (iniciais ou por extenso). Título do artigo: subtítulo. Título da publicação. Disponível em: <endereço eletrônico> Acesso em: data (25 dez. 1999).


Exemplo:

JUNIOR, Lopes; LIMA, Aury Celso de. A prisão de Pinochet e a extraterritorialidade da lei penal. Boletim Paulista de Direito. Disponível em: <http://www.jus.com.br/links/revista.html>. Acesso em: 28 fev. 1999.


CONSTITUIÇÃO FEDEREAL

PAÍS. ESTADO ou MUNICÍPIO. Constituição (data de promulgação). Título. Local: Editor, Ano de publicação. Número de páginas ou volumes. Notas.


Exemplo:

BRASIL. Constituição (1988). Constituição da República Federativa do Brasil. Brasília, DF: Senado, 1988.


LEIS E DECRETOS 

PAÍS, ESTADO ou MUNICÍPIO. Lei ou Decreto, número, data (dia, mês e ano). Ementa. Dados da publicação que publicou a lei ou decreto.


Exemplos:

BRASIL. Lei n. 9610, de 19 de fevereiro de 1998. Dispõe sobre os direitos autorais.

BRASIL. Lei n. 6.895, de 17 de dezembro de 1980. Dá nova redação aos arts. 184 e 186 do Código Penal, aprovado pelo Decreto-lei nº 2.848, de 7 de dezembro de 1940. Diário Oficial da União de 18 de dezembro de 1980.
%-------------------------------------------------------------------------------

\bibliography{reflatex} % geracao automatica das referencias a partir do arquivo reflatex.bib