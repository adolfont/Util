%---------- Primeiro Capitulo ----------
\chapter{Introdu\c{c}\~ao}
\label{chap:intro}

O presente documento \'e um exemplo de uso do estilo de formata\c{c}\~ao \LaTeX\ elaborado para atender \`as Normas para Elabora\c{c}\~ao de Trabalhos Acad\^emicos da UTFPR. O estilo de formata\c{c}\~ao {\ttfamily normas-utf-tex.cls} tem por base o pacote \textsc{abn}\TeX~-- cuja leitura da documenta\c{c}\~ao \cite{abnTeX2009} \'e fortemente sugerida~-- e o estilo de formata\c{c}\~ao \LaTeX\ da UFPR.

Para melhor entendimento do uso do estilo de formata\c{c}\~ao {\ttfamily normas-utf-tex.cls}, aconselha-se que o potencial usu\'ario analise os comandos existentes no arquivo \TeX\ ({\ttfamily modelo\_*.tex}) e os resultados obtidos no arquivo PDF ({\ttfamily modelo\_*.pdf}) depois do processamento pelo software \LaTeX\ + \textsc{Bib}\TeX~\cite{LaTeX2009,BibTeX2009}. Recomenda-se a consulta ao material de refer\^encia do software para a sua correta utiliza\c{c}\~ao~\cite{Lamport1986,Buerger1989,Kopka2003,Mittelbach2004}.

% Trecho extraído do modelo http://www2.dainf.ct.utfpr.edu.br/ppgca/estrutura-academica/regulamentacao/avaliacao/copy_of_ModeloProjetoeSeminarios.doc/view
A redação do plano de pesquisa ou proposta deve refletir o poder de síntese do seu autor. Utilize as formatações de página, espaçamento e fonte aqui apresentados (Fonte Arial 12, espaçamento entre linhas 1,5, folha tamanho A4, margens padrão do Word). 
Deve haver especial atenção com o índice, pois o mesmo é gerado de maneira automática, não devendo ser apagado. Depois de introduzir todos os seus textos sob os itens apropriados, coloque o cursor do mouse sobre a área onde está o índice, clique com o botão direito do mouse, selecione a opção “Atualizar campo” e depois “Atualizar apenas o número das páginas”. Pronto, o Índice indicará as páginas automaticamente. Não é preciso editá-lo. 
O texto de introdução deve conter três tipos de informações: apresentação do problema, estado da arte e justificativa do projeto. Uma vez que nem sempre é clara a linha divisória entre estes três tópicos, optou-se pela construção de uma seção única de introdução que deverá conter todas as informações acima mencionadas, permitindo ao autor elaborar um texto com fluência lógica e sem redundância de informações.
A apresentação ou formulação do problema deve deixar, de forma bem clara, qual será o objeto de estudo do projeto. As razões para a escolha do tema deverão ser justificadas e, para isso, você deverá discorrer sobre a importância do estudo, quais as possíveis repercussões, quais hipóteses a serem verificadas, etc. 
O estado da arte serve para embasar tanto a formulação do problema como sua justificativa. É preciso situar historicamente a evolução do tema, quais as abordagens já investigadas, qual o estágio atual do conhecimento sobre o assunto ou quais as tendências que se apresentam.
A justificativa do projeto deve indicar por que o projeto deve ser feito. Descreva os fatores de motivação que o levaram a abordar e trabalhar no assunto.
As maneiras mais comuns de citações são a indireta e a direta. Na citação indireta, o texto é criado com base na obra de autor consultado, no qual se reproduz o conteúdo e as idéias do documento original. Exemplo, utilizando Sobrenome do Autor (data), quando o nome do autor faz parte do texto: Segundo Souza (1999), a importância do tema [...]. Exemplo, utilizando (SOBRENOME DO AUTOR, ano) quando é citada a síntese de uma informação: A Revolução Industrial modificou definitivamente o cenário urbano (SOUZA, 2001). Na citação direta há a reprodução exata do texto citado entre aspas, como, por exemplo: A justificativa deste comportamento “é resultado da integração entre parasita e hospedeiro, após a conclusão da fase de migração” (SOUZA, 1987). No capítulo Referências, ao final deste documento, há diversos exemplos para apresentação da fonte de uma referência. Dúvidas e maiores detalhes, vide norma ABNT vigente para citações e referências.
Importante: O formato recomendado para as citações e referências é o ABNT. Consulte o orientador para verificar a necessidade de utilização de um formato de citações e de referências diferente, como a norma Vancouver.
Importante: Todos os trabalhos que envolvem seres humanos (inclusive entrevistas) e animais deverão obter aprovação do Comitê de Ética em Pesquisa (CEP) e devem apresentar uma cópia do Termo de Consentimento Informado (TCI), que deve ser assinado pelos participantes da pesquisa. Os requisitos do comitê de ética local estão à disposição no CEP, disponível em http://www.utfpr.edu.br/estrutura-universitaria/pro-reitorias/proppg/comite-de-etica-em-pesquisa-1 . O TCI é o documento em que são informadas aos participantes da pesquisa, em linguagem simples e acessível, todas as implicações possíveis (passadas, presentes ou futuras) da pesquisa para esta pessoa. Ao assinar este termo a pessoa estará autorizando sua inclusão na pesquisa. De modo semelhante, as pesquisas que envolverem animais devem respeitar integralmente os preceitos éticos para experimentação animal. 
O projeto de pesquisa deve estar submetido ao Comitê de Ética em Pesquisa o mais cedo possível, para não inviabilizar os prazos da conclusão do mestrado.
Importante: O Direito Autoral deve ser respeitado. A Constituição da República Federativa do Brasil, em seu Artigo 5, Parágrafo XXVII, indica (BRASIL, 1988): "aos autores pertence o direito exclusivo de utilização, publicação ou reprodução de suas obras, transmissível aos herdeiros pelo tempo que a lei fixar.". A Lei de Direitos Autorais (BRASIL, 1998) afirma:
Art 1º. Esta Lei regula os direitos autorais, entendendo-se sob esta denominação os direitos do autor e os que lhe são conexos.
Art. 7º. São obras intelectuais protegidas as criações de espírito, expressas por qualquer meio ou fixadas em qualquer suporte, tangível ou intangível, conhecido ou que se invente no futuro, como:
I - os textos de obras literárias, artísticas ou científicas;
Art. 22. Pertencem ao autor os direitos morais e patrimoniais sobre a obra que criou.
Art 29. Depende da autorização prévia e expressa do autor a utilização da obra, por quaisquer modalidades, tais como: 
I- a reprodução parcial ou integral;
II- a edição; 
Art. 41. Os direitos patrimoniais do autor perduram por setenta anos contados de 1º de janeiro do ano subsequente ao de seu falecimento, obedecida a ordem sucessória da lei civil.
Art. 46 - Não constitui ofensa aos direitos autorais: 
III - a citação em livros, jornais, revistas ou qualquer outro meio de comunicação, de passagens de qualquer obra, para fins de estudo, crítica ou polêmica, na medida justificada para o fim a atingir, indicando-se o nome do autor e a origem da obra;
A Lei n. 6895, de 17 de dezembro de 1980, que modifica o Código Penal, indica em seu artigo 184 (BRASIL, 1980):
Art. 184 - Violar direito autoral: 
Pena - detenção de 3 meses a 1 ano, ou multa.
§ 1º - Se a violação consistir na reprodução, por qualquer meio, com intuito de lucro, de obra intelectual, no todo ou em parte, sem autorização expressa do autor ou de quem o represente, ou consistir na reprodução de fonograma e videofonograma, sem autorização do produtor ou de quem o represente:
Pena - reclusão de um a quatro anos e multa de Cr\$ 10.000,00 (dez mil cruzeiros) a Cr\$ 50.000,00 (cinqüenta mil cruzeiros).
§ 2º - Na mesma pena do parágrafo anterior incorre quem vende, expõe à venda, aluga, introduz no País, adquire, oculta, empresta, troca ou tem em depósito, com intuito de lucro, original ou cópia de obra intelectual, fonograma ou videofonograma, produzidos com violação de direito autoral.
%-------------------------------------------------------------------------------

\section{Motiva\c{c}\~ao}

Uma das principais vantagens do uso do estilo de formata\c{c}\~ao {\ttfamily normas-utf-tex.cls} para \LaTeX\ \'e a formata\c{c}\~ao \textit{autom\'atica} dos elementos que comp\~oem um documento acad\^emico, tais como capa, folha de rosto, dedicat\'oria, agradecimentos, ep\'igrafe, resumo, abstract, listas de figuras, tabelas, siglas e s\'imbolos, sum\'ario, cap\'itulos, refer\^encias, etc. Outras grandes vantagens do uso do \LaTeX\ para formata\c{c}\~ao de documentos acad\^emicos dizem respeito \`a facilidade de gerenciamento de refer\^encias cruzadas e bibliogr\'aficas, al\'em da formata\c{c}\~ao~-- inclusive de equa\c{c}\~oes  matem\'aticas~-- correta e esteticamente perfeita.

\section{Objetivos}
% Trecho extraído do modelo http://www2.dainf.ct.utfpr.edu.br/ppgca/estrutura-academica/regulamentacao/avaliacao/copy_of_ModeloProjetoeSeminarios.doc/view
Os objetivos devem ser claros, sucintos e diretos. Deve ficar bem evidente qual a pergunta ou questionamento para o qual se busca uma resposta através desta pesquisa. 
Os objetivos são divididos em dois tipos: Objetivo Geral e Objetivos Específicos
%-------------------------------------------------------------------------------

\subsection{Objetivo Geral}
% Trecho extraído do modelo http://www2.dainf.ct.utfpr.edu.br/ppgca/estrutura-academica/regulamentacao/avaliacao/copy_of_ModeloProjetoeSeminarios.doc/view
Como pode ser notado, o título está no singular. Portanto, deve ser apresentado apenas 1 (um) objetivo geral. Aqui deve constar um parágrafo descrevendo esse objetivo.
%-------------------------------------------------------------------------------

Prover um modelo de formata\c{c}\~ao \LaTeX\ que atenda \`as Normas para Elabora\c{c}\~ao de Trabalhos Acad\^emicos da UTFPR~\cite{UTFPR2008} e \`as Normas de Apresenta\c{c}\~ao de Trabalhos Acad\^emicos do DAELN~\cite{DAELN2006}.


\subsection{Objetivos Espec\'ificos}

% Trecho extraído do modelo http://www2.dainf.ct.utfpr.edu.br/ppgca/estrutura-academica/regulamentacao/avaliacao/copy_of_ModeloProjetoeSeminarios.doc/view
O título está no plural. Portanto, espera-se encontrar mais de um objetivo específico neste local. Não confunda objetivo específico com metodologia. Os objetivos específicos são o desdobramento do objetivo geral. 
Pode-se começar esse tópico desta forma: 

Dentre os principais objetivos específicos destacam-se:
Cada objetivo específico será colocado em forma de item e terá uma frase curta, mas que deixe claro qual o objetivo.
A somatória dos objetivos específicos formará o objetivo geral.
%-------------------------------------------------------------------------------

\begin{itemize}
	\item Obter documentos acad\^emicos automaticamente formatados com corre\c{c}\~ao e perfei\c{c}\~ao est\'etica.
	\item Desonerar autores da tediosa tarefa de formatar documentos acad\^emicos, permitindo sua concentra\c{c}\~ao no conte\'udo do mesmo.
	\item Desonerar orientadores e examinadores da tediosa tarefa de conferir a formata\c{c}\~ao de documentos acad\^emicos, permitindo sua concentra\c{c}\~ao no conte\'udo do mesmo.
\end{itemize}

% Trecho extraído do modelo http://www2.dainf.ct.utfpr.edu.br/ppgca/estrutura-academica/regulamentacao/avaliacao/copy_of_ModeloProjetoeSeminarios.doc/view
\section{Estrutura do Trabalho}

Aqui será apresentado a estrutura do trabalho, quantos capítulos e o conteúdo respectivo. O conteúdo de cada capítulo será descrito por uma frase curta e que seja representativa (este item somente será incluído nas versões para qualificação e defesa da dissertação). 
%-------------------------------------------------------------------------------
